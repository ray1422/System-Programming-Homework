\documentclass{ctexart}
\setCJKmainfont{BabelStone Han}
\usepackage{CJKutf8}
\renewcommand{\figurename}{圖}
\usepackage{natbib}
\usepackage{graphicx}
\usepackage{minted}
\usepackage{booktabs}
\usepackage[a4paper, margin=1.5cm]{geometry}
\begin{document}
\pagestyle{headings}
\title{系統程式 作業02}
\author{409410005 鍾天睿}
\date{March 09 2021}
\maketitle
\clearpage

\section{系統環境}
\begin{center}
\begin{tabular}{ l l } 
\hline
發行版 & Arch Linux  \\
\hline
Linux Kernel & 5.10.16-arch1-1  \\
\hline
GCC 版本 & 10.2.0 \\
\hline
Clang 版本 & 11.1.0 \\
\hline
GDB 版本 & 10.1 \\
\hline
CPU & Intel i7-10510U \\
\hline
硬碟 & Samsung PM981 M.2 512 GB PCI Express 3.0 NVMe \\
\hline
\end{tabular}
\end{center}

\section{程式說明}
\subsection{安裝 \& 執行}
輸入指令 \mintinline{bash}{make run} 以編譯及執行自動測試。
\section{實作 File Copy}
在實驗中,測試的目標為利用 hole.c 所生成的帶有 file hole 的檔案測試複製程式。檔案磁碟大小約為 97G,實際大小約為3.8G。程式原始碼在附件資料夾中。
程式執行時間將利用 \mintinline{bash}{time} 測試,並利用 \mintinline{bash}{cmp} 確認檔案內容相同。
\subsection{將讀取檔案讀入 Buffer}
這個章節的實驗將檔案用 \mintinline{c}{read} 讀入 buffer 後再利用 \mintinline{c}{write} 寫入欲寫入的檔案。
\subsection{利用 mmap 映射,並使用 memcpy 複製}
這個章節的實驗將檔案用 \mintinline{c}{mmap} 將檔案內容映射至虛擬位址空間,並使用 \mintinline{c}{memcpy} 複製。
\subsection{實驗結果}
\begin{center}
\begin{tabular}{ l l l l l} 
實驗項目                    & 編譯器    & real (sec)        & user (sec)    & sys (sec)             \\
\toprule
read \& write with buffer   & gcc       & 6.984             & 0.088             & 4.091             \\
                            & clang     & 3.871             & 0.058             & 2.979             \\
\hline
mmap \& memcpy              & gcc       & 4.593             & \textbf{0.462}    & \textbf{2.289}    \\
                            & clang     & \textbf{2.950}    & 0.546             & 2.304             \\
\hline
\end{tabular}
\end{center}
\section{結論}
在本次實驗中發現,使用 \mintinline{c}{mmap} 來實作 file copy 具有比較好的執行時間。
\end{document}
